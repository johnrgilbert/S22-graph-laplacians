\documentclass[11pt]{article}

\usepackage{amsmath}

\newcommand{\m}[1]{{\bf{#1}}}       % for matrices and vectors
\newcommand{\ones}{\m1}             % vector of all ones
\newcommand{\zeros}{\m0}            % vector of all zeros
\newcommand{\tr}{^{\sf T}}          % transpose
\newcommand{\pinv}{^{\dagger}}      % pseudoinverse


\topmargin -.5in
\textheight 8.5in
\oddsidemargin 0pt
\evensidemargin 0pt
\textwidth 6.5in

%-------------------------------------------------------------------------------
\begin{document}
%-------------------------------------------------------------------------------

\title{Homework 2}
\author{CS 292F : Laplacian Matrices and Spectra : Spring 2022}

\date{Out: Wednesday, April 20. Due: Monday, May 2.}
\maketitle

{\bf Homework policy.} 
Please turn in all your homework through GradeScope as a single .pdf file.
For computational problems, turn in listings of any code you wrote; 
any sample output and plots; and an interactive session transcript as
a Matlab diary file, Jupyter notebook .pdf, etc.

You are allowed to discuss the problems in groups of up to three 
(you and up to two others), but you must write up the solutions 
on your own. If you do work with anyone, you must acknowledge your
collaborators. Also, you must cite any references you use other than
the textbook and your class notes.

You are forbidden from searching the web in any way to find answers
to these problems. However, you are welcome to search the web for
advice on using Matlab, Julia, and other software tools. Stackoverflow
is great.

\par\bigskip
{\bf Problem 1. Complete bipartite graph.}
Let $m$ and $n$ be integers.  
Let $K_{m,n}$ be the {\em complete bipartite graph} on $m+n$ vertices. 
(Think of this graph as having $m$ red vertices, $n$ blue vertices,
and all $mn$ possible edges joining a red vertex to a blue vertex.)
What are the eigenvalues of the Laplacian of $K_{m,n}$?
Prove your answer correct.
(Note: You can do this by pure thinking. If it were me, I would do it
by computing the eigenvalues for a few examples in Matlab/Julia/Python, 
guessing the pattern, and proving my answer correct.)

\par\bigskip
{\bf Problem 2. Semidefinite ordering and eigenvalues.} 
We noted in class that if $G\preceq H$ for graphs $G$ and $H$
(or indeed for any two symmetric matrices),
then $\lambda_i(G)\le\lambda_i(H)$ for every $i$.
Show that the converse of this statement is false, 
by finding unweighted graphs $G$ and $H$, both with 
$n$~vertices, for which $\lambda_i(G)\le\lambda_i(H)$ for $1\le i\le n$
but $G\not\preceq H$.
(Hint: I did this by first finding a very simple example consisting
of two small symmetric matrices that were not graph Laplacians,
and then figuring out how to make the same thing happen with an
unweighted Laplacian matrix.)

\par\bigskip
{\bf Problem 3. Inverse and pseudoinverse.}
For this problem, $A$ is an $n$-by-$n$ symmetric matrix with eigenvalues
$\lambda_1\leq\lambda_2\leq\cdots\leq\lambda_n$ 
(possibly including duplicates) 
associated with mutually orthogonal unit-length eigenvectors
$w_1, w_2, \ldots, w_n$.  
As we know, we can write $A$ as a sum over eigenvalues as
$$A = \sum_{i=1}^n \lambda_i w_iw_i^T.$$

\par\medskip
{\bf 3.1.} 
Prove that, if $A$ is nonsingular, its inverse can be written as
$$A^{-1} = \sum_{i=1}^n \frac{1}{\lambda_i} w_iw_i^T.$$

\par\medskip
{\bf 3.2.} 
Whether $A$ is singular or not, its {\em pseudoinverse} is defined as
$$A\pinv = \sum_{\lambda_i\ne 0} \frac{1}{\lambda_i} w_iw_i^T,$$
where the sum is taken over the nonzero eigenvalues of $A$.
If $A$ is nonsingular, then $A\pinv=A^{-1}$.
Prove that if $x$~is orthogonal to the null space of~$A$
(that is, $x^Tw_i=0$ whenever $\lambda_i=0$), then
$$A\pinv A x = AA\pinv x = x.$$

\par\medskip
{\bf 3.3.}
Prove that if $A\succeq 0$ then $A\pinv\succeq 0$.

\par\medskip
{\bf 3.4}
Prove that if $A\succeq B\succeq 0$ and 
$\mbox{null}(A)=\mbox{null}(B)=\mbox{span}(\ones)$,
then $A\pinv \preceq B\pinv$.
(Hint: Take $\ones^T x=0$, let $y=A\pinv x-B\pinv x$, and look at~$y^TBy$.)

\par\bigskip
{\bf Problem 4. Kronecker products and product graphs.}
If $A$ is an $m$-by-$m$ symmetric matrix and $B$ is an $n$-by-$n$ symmetric matrix,
their {\em Kronecker product} is the $mn$-by-$mn$ matrix that
can be written in block form as
$$
A\otimes B = \begin{bmatrix}
  a_{11} B & \cdots & a_{1m} B \\
  \vdots   & \ddots & \vdots   \\
  a_{m1} B & \cdots & a_{mm} B
\end{bmatrix}.
$$
In words, $A\otimes B$ is built by replacing each entry of $A$ with a complete copy of $B$
multiplied by that entry.
(The Kronecker product is defined for nonsymmetric and non-square matrices too,
but we will stick to the square symmetric case here.)

\par\medskip
{\bf 4.1.}
Prove that if $Av=\lambda v$ and $Bw=\nu w$, then $\lambda\nu$ is an eigenvalue 
of $A\otimes B$. What is the corresponding eigenvector (as a function of $v$ and $w$)?

\par\medskip
{\bf 4.2.} 
In class, I defined the product of two graphs $G\times H$, 
using the same notation and definition as Section 6.3 of Spielman.
Spielman mentions that if $G$ has $m$ vertices and $H$ has $n$ vertices, then
$$
L_{G \times H} = (L_G \otimes I_n) + (I_m \otimes L_H).
$$
I actually prefer using this equality as the definition of the graph product,
so I won't ask you to prove it from Spielman's definition.
Use this equality (not Spielman's definition) to prove
that if $\lambda$~is an eigenvalue of $G$ (that is, of $L_G$) and
$\nu$ is an eigenvalue of $H$, then $\lambda+\nu$ is an eigenvalue of~$G\times H$.

\end{document}
