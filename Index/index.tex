\documentclass[11pt]{article}

\usepackage{amsfonts}
\usepackage{amsthm}
\usepackage{amsmath}
\usepackage{graphicx}

\newtheorem{theorem}{Theorem}
\newtheorem*{pathlemma}{Path Lemma}
\newtheorem*{etreelemma}{Etree Fill Lemma}
\newcommand{\m}[1]{{\bf{#1}}}       % for matrices and vectors
\newcommand{\ones}{\m1}             % vector of all ones
\newcommand{\zeros}{\m0}            % vector of all zeros
\newcommand{\diag}{\mbox{diag}}
\newcommand{\Diag}{\mbox{Diag}}
\newcommand{\dilation}{\mbox{dilation}}
\newcommand{\congestion}{\mbox{congestion}}
\newcommand{\stretchh}{\mbox{stretch}}
\newcommand{\Reff}{R^{\mbox{\scriptsize eff}}}  % effective resistance
\newcommand{\pinv}{^{\dagger}}          % pseudoinverse
\newcommand{\sdroot}{^{1/2}}            % semidefinite square root
\newcommand{\pinvroot}{^{\dagger/2}}    % square root of pseudoinverse
\newcommand{\Real}{\mathbb{R}}          % real numbers
\newcommand{\krylov}{\mathcal{K}}       % Krylov subspace
\newcommand{\Tr}{\mbox{Tr}}             % trace of a matrix

\topmargin 0in
\textheight 7.9in
\oddsidemargin 0pt
\evensidemargin 0pt
\textwidth 6.5in

\begin{document}

\title{Index of Notation, Definitions, and Theorems}
\author{CS 292F: Graph Laplacians and Spectra}
\date{Version of March 28, 2022}
\maketitle

There is a lot of variation in terminology and notation in
the field of Laplacian matrix computation and spectral graph 
theory.  
Indeed, even ``Laplacian matrix'' is defined differently by
different authors!

This list gives the versions of notation, terminology, and definitions 
that we will use in CS 292F.
I mostly follow the conventions of Dan Spielman's notes, 
though I prefer not to use greek letters for vectors.
It will also include some key theorems, without proofs.
I will keep adding to this list during the quarter.

\begin{enumerate}

\item
Unless otherwise stated, a {\em graph} $G = (V,E)$ is always 
an undirected graph whose $n$ vertices are the integers 
$1$ through $n$, with no multiple edges or loops.

\item
The {\em degree} of a vertex is the number of edges incident on it, 
or equivalently (because we don't allow multiple edges or loops)
the number of its neighboring vertices.

\item 
A graph is said to be {\em regular} if every vertex has the same degree.

\item
A graph is said to be {\em connected} if, for every choice of two
vertices $a$ and~$b$, there is a {\em path} of edges from $a$ to~$b$.
The {\em connected components} of a graph are its maximal connected
subgraphs.

\item
$K_n$ is the {\em complete graph}, which has $n$ vertices and all $n(n-1)/2$ possible edges.

\item
$P_n$ is the {\em path graph}, which has $n$ vertices and $n-1$ edges in a single path.

\item
$S_n$ is the {\em star graph}, which has $n$ vertices, one with degree $n-1$ and 
$n-1$ with degree 1.

\item
$H_k$ is the {\em hypercube graph}, which has $n=2^k$ vertices, all of degree $k$.
Vertices $i$ and $j$ have an edge between them if $i$ and $j$ differ by a power of 2.
Equivalently, we can identify each vertex with a subset of $\{1,\ldots,k\}$,
with edges to just those subsets formed by adding or deleting one element.

\item
$G_e$ or $G_{(a,b)}$ is the graph with $n$ vertices and only one edge $e = (a,b)$.

\item
We will write a {\em vector} as a lower-case latin letter, 
possibly with a subscript, like $x$ or $w_2$.  
We often think of an $n$-vector as a set of labels for the
$n$ vertices of a graph; 
in that case element $i$ of vector $x$ is written as $x(i)$,
and we may write $x\in\Real^V$ instead of $x\in\Real^n$.
In linear algebraic expressions, vectors are column vectors.

\item
Two special vectors are $\zeros$, the vector of all zeros,
and $\ones$, the vector of all ones.

\item
If $a$ is a vertex then $\ones_a$~is the 
{\em characteristic vector} of~$a$, which
is zero except for $\ones_a(a)=1$.
Similarly if $S$~is a set of vertices, 
then $\ones_S$ is the vector that is equal to one
on the elements of~$S$ and zero elsewhere.

\item
If $x$ and $y$ are vectors of the same dimension, 
$$x^Ty = y^Tx = \sum_{i=0}^{n}x(i)y(i)$$
is their inner product (or dot product).
Thus $\ones^T\!x$ is the sum of the elements of $x$,
and $x^Tx$ is the square of the 2-norm (Euclidean length) of $x$.
If $x^Ty=0$, we call $x$ and $y$ {\em orthogonal}, 
and they are in fact perpendicular as vectors in $\Real^n$.

\item
If $d$ is an $n$-vector, $\Diag(d)$ is the $n$-by-$n$ diagonal 
matrix with the elements of $d$ on the diagonal.
If $A$ is any $n$-by-$n$ matrix, $\diag(A)$ is the $n$-vector
of the diagonal elements of $A$.

\item\label{lap}{\bf Laplacian matrix}.
The Laplacian of graph $G$ is the $n$-by-$n$ matrix $L$
whose diagonal element $L(a,a)$ is the degree of vertex $a$, 
and whose off-diagonal element $L(a,b)$ is~$-1$ if $(a,b) \in E$ 
and $0$ if $(a,b) \notin E$.
This matrix, which we (and Spielman) just call the Laplacian,
is sometimes called the {\em combinatorial Laplacian} to 
distinguish it from the normalized Laplacian below~(\ref{nlap}).
Note that $ L\ones = \zeros$.

\item
$L_e$ or $L_{(a,b)}$ is the $n$-by-$n$ Laplacian matrix
of the graph with $n$ vertices and only one edge $e = (a,b)$.
This matrix has only four nonzero elements, two 1's on the
diagonal and two $-1$'s in positions $(a,b)$ and $(b,a)$;
thus 
$$L_{(a,b)}=(\ones_a-\ones_b)(\ones_a-\ones_b)^T.$$
The Laplacian of any graph $G=(V,E)$ is the sum of the Laplacians
of its edges,
$$L_G = \sum_{e\in E} L_e.$$

\item\label{LQF}{\bf Laplacian quadratic form}.
The {\em Laplacian quadratic form} (or just LQF) is $x^TLx$,
where $L$ is a particular graph's Laplacian and $x$ is a variable $n$-vector.
Its value for a particular vector~$x$ is 
$$x^TLx = \sum_{(a,b)\in E}(x(a)-x(b))^2.$$

\item{\bf Cut vector}.
A cut vector is a vector each of whose elements is $+1$ or $-1$.
We can think of a cut vector~$x$ as representing a {\em cut} that partitions
the vertices of graph into two sets $S = \{\,a : x(a) = 1\,\}$
and $V-S = \{\,a : x(a) = -1\,\}$;
then $x = \ones_S-\ones_{V-S}$.
The LQF evaluated at a cut vector is easily seen to be four times the number 
of edges that cross the cut:
$$ x^TLx = 4\cdot|\{\,(a,b)\in E : a\in S \wedge b\in V-S\,\}|.$$

\item\label{eig}{\bf Eigenvalues and eigenvectors}.
If $Aw=\lambda w$ for any square matrix~$A$, nonzero vector~$w$, and scalar~$\lambda$, 
then $\lambda$ is an {\em eigenvalue} of~$A$ and $w$~is an {\em eigenvector}
associated with~$\lambda$.

\item
If $A$ is square and $B$~is nonsingular, then the eigenvalues of $BAB^{-1}$ are the
same as those of~$A$, 
and the eigenvectors of $BAB^{-1}$ are $B$ times the eigenvectors of~$A$.

\item
Every Laplacian $L$ is {\em positive semidefinite}, 
which (along with symmetry) implies that its $n$ eigenvalues are nonnegative and real.
Zero is an eigenvalue of $L$ with multiplicity
equal to the number of connected components of the graph $G$.  
Therefore,
if $G$ is connected, we have $0 = \lambda_1 < \lambda_2 \leq \cdots \leq \lambda_n$.
In that case the eigenvector $w_1$ is the constant vector $\ones/\sqrt n$.

\item{\bf Fiedler value and Fiedler vector}.
The Fiedler value of a graph is $\lambda_2$, its second-smallest eigenvalue,
and the Fiedler vector is $w_2$, the associated eigenvector.
The Fiedler value of a graph is also called its {\em algebraic connectivity}.
Note that $\lambda_2=0$ iff the graph is not connected.

\item{\bf Orthogonal matrix}.
A square matrix $Q$ is orthogonal if $Q^TQ=I$, that is, its inverse
is its transpose.  
As vectors, 
the columns of $Q$ have unit length and are pairwise perpendicular; 
the same is true of the rows of $Q$.

\item\label{symeig}{\bf Symmetric eigenvalue factorization}.
If the $n$-by-$n$ matrix $A$ is symmetric, then it possesses $n$ real eigenvalues
$\lambda_1\leq\lambda_2\leq\cdots\leq\lambda_n$ (possibly including duplicates) 
associated with $n$ mutually orthogonal unit-length eigenvectors
$w_1, w_2, \ldots, w_n$.  If $W$ is the matrix $[w_1\; w_2\; \ldots\; w_n]$
and $\Lambda$ is the matrix $\Diag(\lambda_1,\ldots,\lambda_n)$
then we can summarize this as $AW = W\Lambda$ and $W^TW=I$.
We also have $A=W\Lambda W^T$, whence
$$A = \sum_{i=1}^n \lambda_i w_iw_i^T.$$

\item{\bf Eigenvector basis}.
If symmetric $A$ and its eigenvalues and eigenvectors are as in~(\ref{symeig}), 
any vector $x$ can be written as a linear combination of eigenvectors,
$$x = \sum_{i=1}^n \alpha_i w_i,$$
where $\alpha_i = w_i^Tx.$  
Multiplication by $A$ acts termwise on such a sum:
$$A^k x = \sum_{i=1}^n \alpha_i\lambda_i^k w_i.$$

\item{\bf Pseudoinverse}.
If $A$ and its eigenvalues and eigenvectors 
are as in~(\ref{symeig}), 
the pseudoinverse of~$A$ is
$$A\pinv = \sum_{\lambda_i\ne 0} \frac{1}{\lambda_i} w_iw_i^T,$$
where the sum is taken over the nonzero eigenvalues of $A$.
If $A$ is nonsingular, $A\pinv=A^{-1}$.
If $x$ is orthogonal to the null space of $A$ 
(i.e.\ $x^Tw_i=0$ whenever $\lambda_i=0$), then
$$A\pinv A x = AA\pinv x = x.$$

\item{\bf Square root}.
The positive semidefinite square root of a positive semidefinite matrix~$A$
with eigenvalues and eigenvectors as in~(\ref{symeig}) is the matrix
$$A\sdroot = \sum_{i=1}^n \lambda_i^{1/2} w_iw_i^T.$$
We write the psd square root of~$A\pinv$ as 
$$A\pinvroot = \sum_{\lambda_i\ne 0} \lambda_i^{-1/2} w_iw_i^T.$$

\item{\bf Trace of a matrix}.
The trace of a square matrix is the sum of its eigenvalues:
$$\Tr(A) = \sum_{i=1}^n \lambda_i.$$
Properties of the trace include:

\begin{itemize}

\item The trace is equal to the sum of the diagonal elements,
$$\Tr(A) = \sum_{i=1}^n A(i,i).$$

\item $\Tr(A+B) = \Tr(A)+\Tr(B)$.

\item $\Tr(AB) = \Tr(BA)$.

\item If $x$ is any vector, $\Tr(Axx^T) = x^TAx$.

\item The trace of the Laplacian of an uweighted graph is twice the
number of edges in the graph. The trace of the Laplacian of a weighted
graph~(\ref{wgraph}) is twice the sum of the edge weights.

\end{itemize}

\item{\bf Rayleigh quotient}.
The Rayleigh quotient of a nonzero vector $x$ and a matrix $A$ is
$$\frac{x^TAx}{x^Tx}.$$
If $Ax=\lambda x$, then the Rayleigh quotient of $x$ and $A$ is $\lambda$.

\item\label{RQT}{\bf Rayleigh quotient theorem}.
The eigenvectors of a symmetric matrix $A$ are critical points of its
Rayleigh quotient (considered as a real-valued function of an $n$-vector).
Specifically, 
$$\lambda_k = \min_{x\perp w_1,\ldots,w_{k-1}} \frac{x^TAx}{x^Tx}
            = \max_{x\perp w_{k+1},\ldots,w_n} \frac{x^TAx}{x^Tx},$$
and the extreme values are attained at $x = w_k$.
In particular, therefore, for a Laplacian $L$ the Fiedler value is
$$\lambda_2 = \min_{\ones^T\!x=0} \frac{x^TAx}{x^Tx},$$
attained at the Fiedler vector $w_2$.

\item\label{CFT}{\bf Courant-Fischer theorem}
(another version of the Rayleigh quotient theorem).
The eigenvalues $\lambda_1\leq\cdots\leq\lambda_n$ of a symmetric matrix $A$ 
are characterized by
$$\lambda_k = \max_{\dim \mathbb{S}=n-k+1}\;\min_{x\in \mathbb{S}}\frac{x^TAx}{x^Tx}
            = \min_{\dim \mathbb{S}=k}\;\max_{x\in \mathbb{S}}\frac{x^TAx}{x^Tx},$$
where $\mathbb{S}$ ranges over subspaces of $\Real^n$.
The extreme values are attained at $x = w_k$.

\item{\bf Test vector}.
A test vector for $\lambda_2$ is an $n$-vector that is
orthogonal to $\ones$.  By the Raleigh quotient theorem,
if $v$ is any test vector then $\lambda_2 \leq v^TLv/v^Tv$.
Note that any vector $x$ can be converted to a test vector 
$v = x - (\ones^T\!x/n) \ones$; in words, subtracting off
the mean of any vector orthogonalizes it against the constant
vector.

\item
The {\em boundary} of a set $S\subseteq V$ of vertices, written $\partial S$,
is the set of edges with just one endpoint in $S$. Formally,
$\partial S = \{\,(a,b)\in E : a \in S \wedge b \in V-S\,\}$.
The number of edges in $\partial S$ is $|\partial S|$.

\item\label{isop}{\bf Isoperimetric ratio}.
The isoperimetric ratio of a set $S\subseteq V$ of vertices,
written $\theta(S)$, is the ratio
$$\theta(S) = \frac{|\partial S|}{|S|}.$$
This is one sort of ``surface-to-volume ratio''; 
see the definition of conductance~(\ref{cond}) 
for another.
The isoperimetric ratio of a graph $G$,
written $\theta_G$, is the smallest isoperimetric ratio over all 
sets with at most half the vertices,
$$\theta_G = \min_{|S|\leq n/2}\theta(S).$$
Note that $\theta_G=0$ if and only if $G$ is not connected.

\item\label{isopthm} {\bf Isoperimetric theorem}.
For any set $S$ of vertices, 
$$\theta(S) \geq \lambda_2(1-|S|/n).$$
It follows that the isoperimetric ratio of the graph is bounded 
in terms of the Fiedler value,
$$\theta_G \geq \lambda_2/2.$$
This says that the larger $\lambda_2$ is, the larger the surface-to-volume
ratio of any relatively small set of vertices must be.

\item\label{cond}{\bf Conductance}.
The conductance of a set $S\subseteq V$ of vertices,
written $\phi(S)$, is the ratio 
$$\phi(S) = \frac{|\partial S|}{\min(d(S),d(V-S))},$$
where $d(S)$ is the sum of the degrees of the vertices in $S$.
This is another sort of ``surface-to-volume ratio'';
isoperimetric number~(\ref{isop}) measures volume just by counting vertices, 
while conductance measures volume by counting vertices weighted by their degrees.
In class we defined conductance for unweighted graphs,
but the definition extends to weighted graphs~(\ref{wgraph}) with
a suitable interpretation of~$d(S)$.
The conductance of a graph $G$,
written $\phi_G$, is the smallest conductance of any nonempty 
proper subset of vertices,
$$\phi_G = \min_{S\subset V}\phi(S).$$
Note that $\phi_G=0$ iff $G$ is not connected.
(``Conductance'' has a different meaning in resistive networks, 
as we'll see later; see~(\ref{resistive}) below.)

\item\label{nlap}{\bf Normalized Laplacian}.
The normalized Laplacian of graph~$G$ is 
the $n$-by-$n$ matrix~$N$
whose diagonal element $N(a,a)$ is equal to 1,
and whose off-diagonal element $N(a,b)$ 
is~$-1/\sqrt{d(a)d(b)}$ if $(a,b) \in E$ and $0$ if $(a,b) \notin E$,
where we define $d$ to be the vector of vertex degrees of~$G$.
Another way to say it is that the normalized Laplacian is 
the (ordinary) Laplacian with rows and columns scaled symmetrically 
to make the diagonal elements equal to~1. 
If $D=\diag(d)$ is the diagonal matrix of degrees, 
then 
$$N = D^{-1/2}LD^{-1/2}.$$
Some authors, including notably Fan Chung in her 
wonderful book {\em Spectral Graph Theory}, 
use the name ``Laplacian'' for this matrix~$N$
instead of for our~$L$.

\item\label{nlapeig}
The normalized Laplacian~$N$ is symmetric and 
positive semidefinite, and like the Laplacian
it has 0 as an eigenvalue with multiplicity
equal to the number of connected components
of~$G$.  In general however $N$'s eigenvalues
and eigenvectors are different from $L$'s.
We write $0=\nu_1\leq \nu_2\leq \cdots\leq \nu_n$
for the eigenvalues of~$N$.
The eigenvector corresponding to~$\nu_1$ is
not the constant vector, but the vector~$d^{1/2}$
of the square roots of the vertex degrees:
$$Nd^{1/2} = D^{-1/2}LD^{-1/2}d^{1/2} = D^{-1/2}L\ones = D^{-1/2}\zeros = \zeros.$$

\item\label{nlaprq}
The Rayleigh quotient for the normalized Laplacian~$N$, 
whose critical points determine the eigenvalues,
is related to a ``generalized Rayleigh quotient'' for the Laplacian~$L$.
Specifically, we have
$$\frac{x^TNx}{x^Tx} = \frac{y^TLy}{y^TDy},$$
where $D=\Diag(d)$ is the diagonal matrix of vertex degrees 
and $y=D^{-1/2}x$.
Thus the eigenvalues of $Nx=\nu x$ come from
the generalized eigenvalue problem $Ly=\nu Dy.$

\item\label{gershgorin}{\bf Gershgorin's theorem}.
If $A$ is any square matrix (real or complex), its $n$~eigenvalues are all contained
in the union of the $n$~disks $D_1,\ldots,D_n$ in the complex plane defined by
$$D_a = \{\alpha : |\alpha-A(a,a)| \leq \sum_{b\neq a}|A(a,b)|\}.$$
This implies, for example, that the largest eigenvalue~$\lambda_n$ of a Laplacian
is at most twice the maximum vertex degree.

\item
It follows from Gershgorin's theorem~(\ref{gershgorin}) that
the eigenvalues of the normalized Laplacian $N$ are always bounded by~0 and~2,
$$0 = \nu_1 \leq \nu_2 \leq \cdots \leq \nu_n \leq 2.$$

\item\label{cheeger} {\bf Cheeger's inequalities}.
The normalized Laplacian can be used to give both 
upper and lower bounds on the conductance,
$$\nu_2/2 \leq \phi_G \leq \sqrt{2\nu_2}.$$
Equivalently,
$$\phi_G^2/2 \leq \nu_2 \leq 2\phi_G.$$
The upper bound on~$\nu_2$ is analogous to the
isoperimetric inequality~(\ref{isopthm}).
The lower bound on~$\nu_2$ is Cheeger's inequality,
one of the most significant theorems of spectral
graph theory.
In class we stated (and mostly proved) these
inequalities for unweighted graphs, but they 
hold for weighted graphs~(\ref{wgraph}) as well; 
the Spielman book proves the weighted version.

\item\label{csi} {\bf Cauchy-Schwarz inequality}.
Just for reference, because it comes up in several of
the proofs we're looking at.  
If $x$ and~$y$ are $n$-vectors, then 
$$|x^Ty| \leq \|x\|\,\|y\|.$$
Equivalently,
$$\Big(\sum_i x(i)y(i)\Big)^2 \leq 
\Big(\sum_i x(i)^2\Big)
\Big(\sum_i y(i)^2\Big).$$

\item\label{wgraph}{\bf Weighted graph}.
A weighted graph is an undirected graph that comes 
with {\em positive} weights on the edges, which we write~$c(e)$ or~$c(a,b)$.
We take $c(a,b)=0$ if $(a,b)$ is not an edge; 
weights on edges are required to be strictly positive.
Note that $c(a,b)=c(b,a)$.
We can think of all graphs as weighted graphs;
an ``unweighted'' graph just has edge weights all equal to~1.

\item
In a weighted graph, we often interpret~$d(a)$, for a vertex~$a$,
not as the number of incident edges but as the sum of the weights
of the incident edges:
$$d(a) = \sum_{b \neq a}c(a,b),$$
and we often define the diagonal matrix $D = \Diag(d)$ as the matrix
whose entries are those sums.
We also (as before) write~$d(S)$, where $S$~is a set of vertices, 
to mean $\sum_{a \in S}d(a)$.

\item\label{wlap}{\bf Weighted Laplacian}.
The Laplacian matrix of a weighted graph
is the $n$-by-$n$ matrix~$L$ whose off-diagonal
element~$L(a,b)$ is~$-c(a,b)$ if $(a,b) \in E$ and $0$ if $(a,b) \notin E$,
and whose diagonal element $L(a,a) = d(a) = \sum_{b \neq a}c(a,b)$ 
is chosen to make the row sums zero.
Like the Laplacian of an unweighted graph, we have $L\ones=\zeros$, 
and indeed 0~is an eigenvalue of~$L$ with multiplicity equal to the 
number of connected components of the graph.
For an unweighted graph, this is equivalent to our previous definition,
with all edge weights equal to~1.

\item\label{wnorm}{\bf Normalized weighted Laplacian}.
The normalized Laplacian matrix of a weighted graph is the matrix~$N$
whose diagonal element~$N(a,a)$ is equal to~1,
and which for each edge~$(a,b)$ has symmetric off-diagonal elements 
$N(a,b) = N(b,a) = -c(a,b)/\sqrt{d(a)d(b)}$.
Here $d(a)$ is the sum of the weights of edges incident on~$a$.
If $D = \Diag(d)$ is the diagonal matrix of those sums
and $L$ is the Laplacian of the weighted graph, then
$$N = D^{-1/2}LD^{-1/2}.$$
Like the normalized Laplacian of an unweighted graph, 
we have $Nd^{1/2}=\zeros$, 
and 0 remains an eigenvalue of~$N$ with multiplicity equal to the 
number of connected components of the graph.
Again this is equivalent to our previous definition for an unweighted graph
if all edge weights are equal to~1.

\item {\bf Multiple of a graph}.
If $G$ is a (weighted) graph and $\alpha>0$ is a constant, $\alpha G$~is the
graph whose edge weights are all multiplied by~$\alpha$.
The ordinary Laplacian of~$\alpha G$ is $\alpha$~times the Laplacian of~$G$,
$$L_{\alpha G} = \alpha L_G.$$ 
On the other hand, the normalized Laplacian of $\alpha G$ is the same as the
normalized Laplacian of~$G$, 
$$N_{\alpha G} = N_G,$$ 
since the normalization wipes out the factor of~$\alpha$.

\item{\bf Krylov subspace}.
The $t$-dimensional Krylov subspace based on a square matrix~$A$ 
and a vector~$b$ is 
$$\krylov_t(A,b) = \mbox{span}(b,Ab,A^2b,\ldots,A^{t-1}b).$$

\item\label{QR}{\bf Symmetric $QR$ algorithm}.
This algorithm computes all the eigenvalues and all the eigenvectors
of a symmetric matrix~$A$ in two phases. 
The first phase applies a sequence of $n-2$ elementary orthogonal 
transformations called {\em Householder reflections} to~$A$,
symmetrically from the left and right.
The reflections are chosen to zero out each column in turn 
below its first subdiagonal element (and, because of symmetry,
each row after its first superdiagonal element), 
giving the factorization
$$Q^TAQ = T,$$
where $Q$ is orthogonal and $T$ is tridiagonal (and symmetric).
The second phase is iterative, and converts $T$ to a diagonal matrix
by applying a sequence of elementary orthogonal transformations called
{\em Givens rotations} to reduce the magnitude of the off-diagonal
elements by a process called ``bulge-chasing'' that is 
reminiscent of squeezing the last bit of toothpaste from a tube.
Iterations continue until the off-diagonal elements are sufficiently small
to be negligible.
This gives the factorization
$$V^TTV = \Lambda,$$
where $V$ is orthogonal and $\Lambda$ is diagonal.
Taking $W=QV$, we then have the eigenvalue factorization
$$A = W\Lambda W^T.$$
The first phase does $O(n^3)$ work, and we can think of the second phase
(modulo details about convergence and floating-point arithmetic)
as doing $O(n^2)$ work. 
This is the workhorse method for computing all eigenvalues and eigenvectors
of dense matrices, 
but it can't be used for very large sparse matrices
both because of the $n^3$~work and because the first phase 
needs $n^2$~memory to store intermediate fill.
See the Demmel or Trefethen/Bau textbooks for details.

\item{\bf Lanczos iteration}.
The Lanczos iteration computes the matrices $Q$ and $T$ above,
one column at a time.
It begins with an arbitrary (typically random)
unit vector $q_1$ as the first column
of $Q$, and at step $t$ it computes the vector $v=Aq_t$,
orthogonalizes $v$ against columns $q_t$ and $q_{t-1}$,
and scales the orthogonalized vector to unit length.
Because $T$ is symmetric and tridiagonal, 
the resulting vector is actually orthogonal to all columns $q_1$ 
through $q_t$, and it becomes column $q_{t+1}$.
The coefficients of the orthogonalization become the entries of $T$.
See the class notes (or Demmel or Trefethen/Bau) for the details
of the formulas.
The Lanczos iteration can be viewed as building orthogonal bases 
for the Krylov subspaces $K_t(A,q_1)$.
Unlike the $QR$ algorithm~(\ref{QR}),
the only thing Lanczos needs to do with
the matrix $A$ is to multiply it by vectors, 
which is useful when $A$ is sparse.

\item{\bf Lanczos algorithm}.
The Lanczos algorithm computes approximations
to some of the eigenvalues of~$A$. 
It first performs some number~$t$ (typically much less than~$n$)
of Lanczos iterations to get an $n$-by-$t$ matrix~$Q_t$ 
with orthonormal columns (i.e.\ $Q_t^TQ_t = I_t$) and a 
$t$-by-$t$ symmetric tridiagonal matrix~$T_t = Q_t^TAQ_t$.
It then uses the second (``toothpaste-squeezing'') phase of
the symmetric $QR$ algorithm~(\ref{QR}) to diagonalize
$T_t=V_t\Theta V_t^T$, where $V_t$ is $t$-by-$t$ and orthogonal,
and $\Theta$ is diagonal.
Then the numbers $\theta_1,\theta_2\ldots,\theta_t$ on the diagonal
of~$\Theta$ (the eigenvalues of~$T_k$) are {\em Ritz values} for~$A$,
and the columns of~$Q_kV_k$ are the corresponding {\em Ritz vectors}.
(There is a lot of numerical-algorithm engineering involved in a
practical implementation of Lanczos; 
see Demmel or the 1994 Grimes/Lewis/Simon SIMAX paper for details.)

\item{\bf Ritz values}.
The Ritz values of a matrix approximate some of its eigenvalues.
The whole story is rather involved; see Demmel for more of it.
In exact real arithmetic, if $A$ has no multiple eigenvalues, 
the Ritz values after $t=n$~steps are equal to the $n$~eigenvalues 
of~$A$. 
Multiple (or very close) eigenvalues are tricky; 
block versions of Lanczos can be used here.
In exact arithmetic, at stage~$t$, every Ritz value
is guaranteed to be within~$\beta_t$ of some eigenvalue,
where $\beta_t$ is the subdiagonal element in column~$t$ of~$T$.
Generally speaking, the extreme Ritz values 
(those closest to $+\infty$ and~$-\infty$)
tend to converge quickly to the extreme eigenvalues.

\item\label{schur}{\bf Schur complement}.
If matrix $M$ is partitioned into a 2-by-2 block form
$$ M =
\begin{pmatrix}
A & B \cr
C & D \cr
\end{pmatrix}
$$
such that $A$ is square and nonsingular, 
then the Schur complement of $A$ in $M$
(sometimes also called the ``Schur complement on $D$'')
is the matrix 
$$D - CA^{-1}B.$$

\item\label{inc}{\bf Incidence matrix}.
If $G$ is a graph with $n$~vertices and $m$~edges,
an incidence matrix of~$G$ is an $n$-by-$m$ matrix~$U$
with a column for each edge of~$G$.  
The column for edge~$(a,b)$ contains two nonzeros,
a $1$ and a~$-1$, one in row~$a$ and one in row~$b$.
The incidence matrix is not unique; permuting its columns
or negating some of its columns produces another incidence matrix.
Any incidence matrix~$U$ is related to the Laplacian~$L_G$ by
$$L_G = UU^T.$$

\item
If $G$ is a weighted graph, its incidence matrix is the same
as above~(\ref{inc}), without weights.  If $C=\diag(c)$ is
the diagonal matrix of edge weights, in the same order as the
columns of~$U$, then the weighted Laplacian~$L_G$ satisfies
$$L_G = UCU^T.$$

\item\label{aug}{\bf Augmented matrix}.
An augmented matrix of an unweighted graph with $n$~vertices 
and $m$~edges is a
symmetric $(n+m)$-by-$(n+m)$ matrix defined in block form as
$$
\begin{pmatrix}
I & U^T \cr
U & 0 \cr
\end{pmatrix}
,
$$
where $U$ is an incidence matrix and $I$ is the $m$-by-$m$ 
identity matrix.
An augmented matrix of a weighted graph is
$$
\begin{pmatrix}
R & U^T \cr
U & 0 \cr
\end{pmatrix}
,
$$
where $R=\diag(1/c)=C^{-1}$ is the diagonal matrix of inverse edge weights.
The Schur complement of $R$ is then $-UR^{-1}U^T = -L$, the negative
of the weighted Laplacian.

\item\label{resistive}{\bf Resistive network}.
A resistive network is a weighted graph with $n$~vertices
interpreted as nodes of an electrical circuit and $m$ edges interpreted
as resistors joining pairs of nodes.  
If the resistor at edge $e$ has resistance $r(e)$, the edge's weight
is the inverse resistance $c(e) = 1/r(e)$.
(Inverse resistance is often called ``conductance,'' whence the letter $c$, 
but we will not use the term in this context to avoid confusion with the
unrelated notion of graph conductance in~(\ref{cond}) above.)

\item
In a resistive network~$G$, suppose a current $i(a)$ is injected at each
vertex~$a$, where $\ones^T i=0$ so the same total current is injected and
removed from the network as a whole.
For each edge $(a,b)\in E$ with $a<b$, let $f(a,b)$ be the {\em current} or 
{\em flow} along edge $(a,b)$.
Then {\em Ohm's law}, or current times resistance equals voltage, says
$$v(a)-v(b) = f(a,b)r(a,b) \;\;\;\; \mbox{for all $(a,b)\in E$}.$$
{\em Kirchoff's current law}, or current entering a node equals current leaving the node,
says
$$\sum_{b:(a,b)\in E} f(a,b) = i(a) \;\;\;\; \mbox{for all $a\in V$}.$$
The two laws can be combined in one augmented system (\ref{aug}) as
$$
\begin{pmatrix}
R & U^T \cr
U & 0 \cr
\end{pmatrix}
\begin{pmatrix}
f \cr
-v \cr
\end{pmatrix}
=
\begin{pmatrix}
0 \cr
i \cr
\end{pmatrix},
$$
where $U$ is the incidence matrix of~$G$
with appropriate signs, and $R$ is the
$m$-by-$m$ diagonal matrix of edge resistances.
The Schur complement of $R$ is $-UR^{-1}U^T$, 
which is the negative Laplacian $-L$.  
A step of block Gaussian elimination on the augmented
system thus leads to the Laplacian linear system
$$Lv=i$$
relating the node voltages to the externally injected currents.

\item{\bf Effective resistance}.
In a resistive network, the effective resistance between
two vertices~$a$ and~$b$, written $\Reff(a,b)$, 
is the positive difference in voltage
between~$a$ and~$b$ when one unit of current is injected at~$a$
and extracted at~$b$.
That is, if $Lv=\ones_a-\ones_b$ for~$a<b$, then
$$\Reff(a,b) = (\ones_a-\ones_b)^Tv 
  = (\ones_a-\ones_b)^TL\pinv (\ones_a-\ones_b) = v^TLv,$$
where $\ones_a-\ones_b$
is the vector whose $k$'th element is equal to 1 when $k=a$, 
equal to $-1$ when $k=b$, and equal to zero elsewhere.
We write $\Reff_G(a,b)$ if the graph is not clear from context.

\item\label{GE}{\bf Symmetric Gaussian elimination}.
If $L$ is the Laplacian of a connected weighted graph~$G$ 
with $n>1$~vertices,
then the result of one step of Gaussian elimination on~$L$ is the
$n-1$-by-$n-1$ matrix
$$L_B = L(2:n, 2:n) - \frac{1}{L(1,1)}L(2:n,1)L(1,2:n),$$
which is a rank-1 modification to the submatrix $L(2:n, 2:n)$.
Note that $L_B$ is the Schur complement of~$L(1,1)$ in~$L$.
Also, $L_B$ is the Laplacian matrix of the (suitably weighted) 
graph obtained from~$G$
by adding edges between all neighbors of vertex~1,
and then deleting vertex~1 and all its incident edges.

If $B$ is any set of vertices of~$G$ and $A=V-B$, 
and Gaussian elimination is used to eliminate all the 
vertices of~$A$ in any order, the result is the matrix
$$L_B = L(B,B) - L(B,A)L(A,A)^{-1}L(A,B),$$
which is the Schur complement of~$L(A,A)$ (or on~$L(B,B)$) in~$L$.
Then $L_B$ is the Laplacian matrix of the (suitably weighted)
graph obtained from~$G$ by adding an edge between every pair 
of vertices in~$B$ that are connected in~$G$ by a path of 
vertices in~$A$.

\item{\bf Equivalent networks}.
If $G$ is a connected weighted graph whose vertices are
partitioned into a set~$B$ of ``boundary'' vertices and
a set~$V-B$ of ``interior'' vertices, 
and if $L_B$ is the Schur complement on~$L_G(B,B)$ 
as above~(\ref{GE}),
then the effective resistances between vertices of~$B$
are the same in~$L_G$ and in~$L_B$:
$$\Reff_{L_B}(a,b) = \Reff_{L_G}(a,b)\;\;\mbox{for all $a,b\in B$}.$$
Note that $L_B$ is {\em not} a submatrix of~$L_G$, 
and the graph of~$L_B$ generally has more edges than the induced 
subgraph of~$G$ on vertices~$B$.

\item\label{Cholesky}{\bf Cholesky factorization}.
If~$A$ is any positive definite matrix 
(or, with some care, a positive semidefinite matrix),
the Cholesky factorization is $A=R^TR$, where $R$
is an upper triangular matrix with positive diagonal elements
(non-negative in the semidefinite case).
The Cholesky factorization of any $n$-by-$n$ matrix can
be computed in $O(n^3)$ time and $O(n^2)$ memory;
some but by no means all sparse matrices have better
bounds.

\item{\bf Cholesky graph game}.
Given positive (semi)definite~$A$ with undirected graph~$G(A)$,
the undirected graph~$G^+(A)=G(R+R^T)$ of the Cholesky factors
of~$A$ is obtained as follows:
\begin{verbatim}
        for a = 1 : n
          mark vertex a;
          add "fill" edges between the umarked neighbors of vertex a;
        end for
\end{verbatim}
(This gives the nonzero structure of~$R$ but not the nonzero values.)
We are free to mark the vertices in any order;
choosing a different order corresponds to applying a
permutation symmetrically to the rows and columns of~$A$.

\item{\bf Parter's theorem}.
If the graph $G(A)$ is a tree, a vertex ordering exists for
which the Cholesky factorization adds no fill and solving
$Ax=b$ takes only $O(n)$ time and memory.

\item{\bf Nested dissection}.
If the graph $G(A)$ is the $\sqrt{n}$-by-$\sqrt{n}$ grid graph,
the best possible elimination ordering has $O(n\log n)$ fill,
for which Cholesky takes $O(n^{3/2})$ time.
The same upper bounds hold for any planar graph.
For the three-dimensional grid graph, the best possible
fill is $O(n^{4/3})$ and Cholesky takes $O(n^2)$ time.

\item{\bf Matrix polynomials}.
If $p(z) = \sum_{k=0}^t \beta_kz^k$ is a polynomial in a 
scalar variable~$z$, and $A$~is a matrix,
we write $p(A) = \sum_{k=0}^t \beta_kA^k$.
Note that every vector in the Krylov subspace $\krylov_{t+1}(A,b)$
is~$p(A)b$ for some polynomial~$p(z)$ of degree~$t$.

\item
If $A$ is a symmetric matrix, $Ax=b$, 
and $q(z) = 1 - zp(z)$ is a polynomial with
$q(0)=1$ and $q(\lambda)=0$ for every eigenvalue~$\lambda$ of~$A$,
then $p(A)b = x$.
Therefore $x$ is in the Krylov subspace $\krylov_{t}(A,b)$, 
where $t$ is the number of {\em distinct} eigenvalues of~$A$.

\item{\bf Condition number}.
The condition number of a square matrix~$A$ is 
$\kappa(A)=\|A\|\,\|A^{-1}\|$, interpreted as~$\infty$
if $A$ is singular.  If $A$ is symmetric and positive 
definite, $\kappa(A) = \lambda_n/\lambda_1$ is the ratio
of the extreme eigenvalues.

\item{\bf Finite condition number}.
For a symmetric positive semidefinite matrix~$A$,
the finite condition number
$\kappa_f(A)=\lambda_n/\lambda_k$, where $\lambda_k$ is
the smallest nonzero eigenvalue.

\item
Let~$A$ and~$B$ be symmetric positive semidefinite matrices
with the same null space (e.g., weighted Laplacians of two
connected graphs on the same vertices).
The {\em finite condition number}~$\kappa_f(A,B)$ is $\kappa_f(AB\pinv)$.
Note that $\kappa_f(A,B) = \kappa_f(B,A)$.

\item{\bf Conjugate gradient}.\label{cg} 
The conjugate gradient algorithm (or CG)
solves $Ax=b$, where $A$ is a symmetric, positive definite
matrix (see Shewchuk for details).
Each iteration performs one matrix-vector multiplication with~$A$
and some vector arithmetic, taking $O(n+m)$~time per iteration
if $A$~has $m$~nonzeros.
The relative error in the approximate solution~$x_t$ is bounded by
$$\frac{\|x_t-x\|_A}{\|x\|_A} < \epsilon$$
after 
$$t = O(\sqrt{\kappa(A)}\,\log(1/\epsilon))$$
iterations (in exact arithmetic), 
where $\kappa(A)=\lambda_n/\lambda_1$ is the condition number of~$A$
and $\|v\|_A = (v^TAv)^{1/2}$ is the $A$-norm.
With some care, CG can also be used for a positive semidefinite
matrix whose null space is known, e.g.\ a weighted graph Laplacian.
In that case, 
the convergence bound uses the finite condition number~$\kappa_f(A)$.

\item\label{evalcluster}
When some of the eigenvalues of~$A$ are clustered together, 
better bounds than $O(\sqrt{\kappa_f(A)}\,\log(1/\epsilon))$
may hold on the number of CG iterations to convergence.
For example, if all but $k$ of the nonzero eigenvalues are
contained in a range $a\le\lambda_i\le b$, CG will converge to
$$\frac{\|x_t-x\|_A}{\|x\|_A} < \epsilon$$
after at most
$$t = k + O(\sqrt{b/a}\,\log(1/\epsilon))$$
iterations (in exact arithmetic).

\item{\bf Preconditioned conjugate gradient}.\label{pcg} 
The preconditioned conjugate gradient algorithm (or PCG)
solves $Ax=b$ by applying CG to the linear system 
$$(B^{-1/2}AB^{-1/2})(B^{1/2}x) = B^{-1/2}b,$$
where $A$ and~$B$ are symmetric positive definite.
Each iteration of PCG performs one matrix-vector multiplication with~$A$,
one linear system solve with~$B$, and some vector arithmetic.
Matrix~$B$ is called a {\em preconditioner} for~$A$, 
and may or may not be formed explicitly.
A good preconditioner satisfies two criteria:
\begin{itemize}
\item It is easier to solve linear systems with $B$ than with $A$.
\item The condition number 
$\kappa(B^{-1/2}AB^{-1/2}) = \kappa(AB^{-1}) = \kappa(A,B)$
is smaller than $\kappa(A)$.
\end{itemize}
With some care, PCG can also be used with positive semidefinite
matrices $A$ and~$B$ if they have the same null space.
In that case, 
the relevant condition number is the finite condition number~$\kappa_f(A,B)$.

\item{\bf Trace-based bounds on CG}.
A consequence of~(\ref{evalcluster}) is the following bound on
conjugate gradient iterations for a matrix (or preconditioned matrix)~$A$
whose smallest nonzero eigenvalue is known to be at least~1.
In that case the number of iterations to $\epsilon$-convergence
is at most $O(\Tr(A)^{1/3}\log(1/\epsilon))$.

\item {\bf Semidefinite order (Loewner order)}.
If $A$ is a symmetric matrix, 
$A\succeq 0$ means that $A$ is positive semidefinite 
(all eigenvalues of~$A$ are non-negative) and
$A\succ 0$ means that $A$ is positive definite
(all eigenvalues of~$A$ are positive).
If $A$ and~$B$ are symmetric matrices, 
$A\preceq B$ means $B-A\succeq 0$ and
$A\prec B$ means $B-A\succ 0$.
If $G$ and~$H$ are graphs or weighted graphs, 
$G\preceq H$ means $L_G\preceq L_H$.
Here are some properties of this order.

\begin{itemize}

\item Loewner order is only a partial order;
there are matrices $A\succeq 0$ and~$B\succeq 0$
for which neither $A\preceq B$ nor~$B\preceq A$.

\item If $A\preceq B$ and $B\preceq C$, then $A\preceq C$.

\item If $A\preceq B$ and $B\preceq A$, then $A=B$.

\item $A\preceq B$ iff $x^TAx \leq x^TBx$ for all vectors~$x$.

\item $A\preceq B$ implies $\lambda_i(A) \leq \lambda_i(B)$
for all $1\leq i\leq n$. The converse is false.

\item $A\preceq B$ implies $CAC^T \preceq CBC^T$ for all matrices~$C$, 
including non-symmetric and non-square matrices. The converse holds
for nonsingular matrices~$C$.

\item If $0\preceq A\preceq B$, then $B\pinvroot A B\pinvroot \preceq I$.

\item If $0\preceq A\preceq B$ and $A$ and $B$ have the same null space,
then $B\pinv\preceq A\pinv$.

\item
Let $G$ and $H$ be resistive networks on the same number of vertices.
If $H\preceq G$, then for every pair of vertices $a$,~$b$ we have
$\Reff_G(a,b) \leq \Reff_H(a,b)$.

\item
If~$A\succeq 0$ and~$B\succeq 0$ have the same null space,
and $\alpha B \preceq A \preceq \beta B$,
then $\alpha \leq \lambda \leq \beta$
for every nonzero eigenvalue~$\lambda$ of~$AB\pinv$,
and therefore $\kappa_f(A,B) \leq \beta/\alpha$.

\end{itemize}

\item {\bf Graph approximation}.
For any constant $\alpha\geq 1$, 
(weighted) graph~$H$ is an {\em $\epsilon$-approximation} 
of (weighted)~graph $G$ if $(1-\epsilon) H \preceq G \preceq (1+\epsilon)H$.
This definition actually applies to all symmetric matrices, 
not just graph Laplacians.
We say loosely that~$G$ and~$H$ are $\epsilon$-approximations of each other,
though the values of~$\epsilon$ are slightly different in each direction.

This is a powerful statement.
If $G$ and~$H$ are $\epsilon$-approximations: 

\begin{itemize}

\item Their eigenvalues are similar: 
$(1-\epsilon)\lambda_i(H)\leq\lambda_i(G)\leq(1+\epsilon)\lambda_i(H)$
for all~$i$.

\item They operate similarly on vectors:
$\|Gx-Hx\|/(\|G\|\,\|x\|) \leq \epsilon$.

\item Solutions of linear systems are similar:
$\|G\pinv b - H\pinv b\|/(\|G\pinv\|\,\|b\|) \leq \epsilon/(1-\epsilon)$.

\item They are good preconditioners for each other:
$\kappa_f(G,H) = 1 + O(\epsilon)$.

\item Their effective resistances $\Reff_G(a,b)$ and $\Reff_H(a,b)$
are similar for every pair $a$, $b$ of vertices.

\item All their weighted cuts are similar, 
since $\ones_A^TG\pinv\ones_A$ is close to $\ones_A^TH\pinv\ones_A$ 
for every set~$A$ of vertices.

\end{itemize}

\item{\bf Leverage score}.
If $G$ is a graph with edge weights~$c(e) = c(a,b)$, 
the leverage score of edge~$e = (a,b)$ is
$$\ell_e=c(a,b)\Reff(a,b).$$
Since, in a resistive network, 
$c(e)$ is the inverse of the resistance  of edge $e$,
the leverage score is the ratio of the effective resistance
across the edge to the edge resistance. 
Therefore $0\leq\ell_e\leq 1$ for every edge.
If cutting edge~$e$ disconnects the graph, then $\ell_e=1$.
The sum of the leverage scores of all edges is $n-1$;
see~(\ref{rtree}).

\item{\bf Spanning trees, fundamental cycles}.\label{stree}
If $G$ is a connected graph (or a connected weighted Laplacian),
a spanning tree of~$G$ is a subgraph~$T$ of~$G$ that has
no cycles (a tree) and includes all the vertices of~$G$ (spanning).
Unless otherwise stated, 
edges of~$T$ have the same weight in~$T$ as they have in~$G$.
Each edge~$e=(a,b)$ that is in~$G$ but not in~$T$ induces a
fundamental cycle consisting of~$e$ and the unique path~$P(e)$
in~$T$ between its endpoints~$a$ and~$b$.
For an edge $e=(a,b)$ of~$T$, we write~$P(e)=(a,b)$ 
for the length-one path between its endpoints, 
but~$e$ does not induce a fundamental cycle.

\item{\bf Random spanning tree theorem}.\label{rtree}
If we choose a spanning tree~$T$ of connected~$G$ at random 
with probability proportional to the product of its edge weights, 
then for every edge~$e\in G$,
$$\Pr[e\in T] = \ell_e.$$
It follows that the sum of the leverage scores is $n-1$,
the number of edges in a spanning tree.

\item{\bf Spectral sparsifiers}.
Let $G$ be a connected weighted Laplacian.
Select a subgraph~$H$ at random by choosing each edge
of~$G$ independently with probability proportional to its leverage score.
Then inflate the weight of each edge of~$H$ 
by the inverse of its probability of being chosen.
This inflation makes the expected value of~$H$ (as a matrix) equal to~$G$.

If we take the constant of proportionality to be $\tau=3.5\ln n/\epsilon^2$,
we choose edge~$e$ with probability~$\tau\ell_e$.
If $e$ is chosen then its new weight in~$H$ 
becomes~$c(e)/\tau\ell_e = 1/\tau\Reff(e)$.
It is a theorem that (with high probability) this subgraph~$H$ 
is an $\epsilon$-approximation of~$G$ and has $O(n\log n/\epsilon^2)$ edges.
A more complicated construction yields an $\epsilon$-approximation
with only $O(n/\epsilon^2)$ edges.

\item{\bf Stretch of an edge}.
If $T$ is a spanning tree of~$G$ 
(connected Laplacians with edge weights $c(e)$),
the stretch of an edge~$e$ of~$G$ is the following sum along the
tree path $P(e)$ between its endpoints:
$$\stretchh(e) = c(e)\sum_{f\in P(e)}\frac{1}{c(f)}.$$
If we interpret $G$ and~$T$ as resistive networks
with resistances $r(e) = 1/c(e)$, 
then the stretch of edge~$e$ is just~$r(P(e))/r(e)$,
which is ratio of the effective resistance in $T$ to the
edge resistance in~$G$ between the endpoints of~$e$,
$$\stretchh(a,b) = \frac{\Reff_T(a,b)}{r_G(a,b)}.$$
We could use the latter equality as the definition of stretch
for an arbitrary subgraph, not necessarily a tree.

\item{\bf Stretch of a tree}.
The stretch of a spanning tree~$T$ of~$G$ 
(connected weighted Laplacians)
is the total stretch of the edges of~$G$ (not the edges of~$T$),
$$\stretchh_G(T)=\sum_{e\in G}\stretchh(e)=\Tr(T\pinv G).$$
The second equality above states that the total stretch is equal to 
the trace (or sum of eigenvalues) of the preconditioned matrix,
which is straightforward to verify.

\item{\bf Low-stretch trees}.
A nontrivial theorem is that every weighted graph~$G$ 
with $n$~vertices and $m$~edges
has a spanning tree~$T$ with nearly linear stretch,
$$\stretchh_G(T)=O(m\log n\log\log n),$$
which can be found in $O(m\log n\log\log n)$ (expected) time.
It is an open problem whether the stretch can be improved
to $O(m\log n)$ in general, which would be best possible.

\item{\bf Preconditioning by trees}.
When a low-stretch spanning tree is used to precondition 
a connected Laplacian linear system, the number of iterations to
$\epsilon$-convergence is $\tilde{O}(m^{1/3}\log(1/\epsilon))$ 
and the running time is $\tilde{O}(m^{4/3}\log(1/\epsilon))$.
Here $m$ is the number of edges in the graph, 
and $\tilde{O}()$ hides factors polynomial in~$\log n$.

\item {\bf Spielman-Teng solver}.
In a series of papers published from 2004 to~2014,
Spielman and Teng used spectral sparsification and
recursive preconditioning to give the first algorithm that
solves Laplacian linear systems in time $\tilde{O}(m\log(1/\epsilon))$.

\item {\bf Incremental sparsifiers}.
Koutis, Miller, and Peng (2010) presented a randomized algorithm that,
given a weighted graph~$G$ and a constant~$3\leq\kappa<3m$, 
produces a so-called incremental sparsifier, 
which is a graph~$H$ with $n-1+ \tilde{O}(m/\kappa)$ edges 
for which $G\preceq H\preceq \kappa G$.
The algorithm samples edges of~$G$ with probability proportional 
to their stretch relative to a low-stretch spanning tree.
It runs in time~$\tilde{O}(m)$.


\item {\bf KMP solver}.
Koutis, Miller, and Peng (2010) used incremental sparsifiers and
recursive preconditioning to give an algorithm that solves Laplacian 
linear systems in time~$\tilde{O}(m\log(1/\epsilon))$.
The KMP algorithm is closely related to the Spielman-Teng algorithm,
but is simpler and also faster;
in S-T the $\tilde{O}()$ hides a high power of~$\log n$,
while in KMP it hides only $\log^2 n$ times a power of~$\log\log n$.

\item {\bf Other fast solvers.}
Since KMP, several other $\tilde{O}(m\log(1/\epsilon))$-time
Laplacian solvers have been invented, 
and the field is still seeing rapid development.
Two that are particularly simple in structure are
Kelner et al.'s 2013 algorithm that performs random Kaczmarz 
projections against fundamental cycles with respect to a
low-stretch tree,
and Kyng and Sachdeva's 2016 algorithm that produces an
incomplete Cholesky factor by randomly sampling fill edges
to keep at each elimination step.
As of this writing, the asymptotically most efficient
is due to Cohen et al.\ (2014) 
and runs in time $O(m\log^{1/2}n\log\log^c n\log(1/\epsilon))$.

\end{enumerate}

\end{document}
